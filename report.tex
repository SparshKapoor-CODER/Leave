\documentclass[12pt,a4paper]{article}

\usepackage{geometry}
\geometry{margin=1in}

\usepackage{graphicx}
\usepackage{amsmath}
\usepackage{amssymb}
\usepackage{hyperref}
\usepackage{listings}
\usepackage{xcolor}
\usepackage{float}
\usepackage{setspace}
\usepackage{tabularx}
\usepackage{longtable}

\onehalfspacing

\hypersetup{
    colorlinks=true,
    linkcolor=black,
    urlcolor=blue
}

\lstset{
  basicstyle=\ttfamily\small,
  breaklines=true,
  frame=single,
  backgroundcolor=\color{gray!10},
  keywordstyle=\color{blue},
  commentstyle=\color{gray!70},
  stringstyle=\color{teal!70!black}
}

\title{
\textbf{Smart QR-Based Hostel Leave Verification System} \\
\vspace{0.4cm}
\large Phase I: Problem Analysis, System Design, and Feasibility Study
}

\author{Sparsh Kapoor}
\date{\today}

\begin{document}

\maketitle
\thispagestyle{empty}
\newpage

\tableofcontents
\newpage

%-------------------------------------------------
\section{Introduction}
University hostels operate as high-density residential environments where administrative efficiency, security, and student welfare must coexist. A critical component of hostel administration is the regulation of student movement through formal leave approvals. \\
\noindent At present, the hostel leave verification process in many universities, including VIT Bhopal University, remains largely manual at the point of exit. While digital systems such as VTOP handle leave applications and approvals, the final authorization still depends on human verification using screenshots or visual confirmation. \\
\noindent This disconnect between digital approval and physical execution introduces inefficiencies, governance gaps, and operational risks. The proposed project addresses this gap by introducing a secure, automated, and auditable QR-based hostel leave verification system tightly integrated with the existing leave-management workflow implemented in this Flask application.

%-------------------------------------------------
\section{Background and Context}
The hostel leave process is designed to ensure:
\begin{itemize}
    \item Student safety and accountability
    \item Compliance with university policies
    \item Parental assurance
    \item Controlled campus movement
\end{itemize}
However, increasing student strength, limited supervisory staff, and peak-period congestion have made manual verification unsustainable. What was once manageable with small student populations has now become a recurring administrative and operational challenge.

%-------------------------------------------------
\section{Detailed Problem Analysis}
\subsection{Human Dependency and Administrative Bottlenecks}
The current system places excessive operational responsibility on hostel supervisors, who must manually verify approvals for hundreds of students within limited time windows. This leads to:
\begin{itemize}
    \item Delays in issuing permit slips
    \item Queue congestion at hostel offices
    \item Increased stress for both staff and students
\end{itemize}
During peak hours such as weekends, holidays, or emergency travel periods, this dependency becomes a critical bottleneck.

\subsection{Rude Behavior and Student Grievances}
Recent observations indicate frequent complaints regarding:
\begin{itemize}
    \item Rude or dismissive behavior by supervisors
    \item Arbitrary delays despite valid approvals
    \item Inconsistent treatment of students
\end{itemize}
Such experiences negatively impact student mental well-being and contribute to an adversarial environment between students and hostel administration.

\subsection{Missed Travel and Escalations}
Due to delays or discretionary denial at the final verification stage:
\begin{itemize}
    \item Students miss buses, trains, and flights
    \item Emergency leaves are delayed
    \item Matters escalate to higher hostel management
\end{itemize}
These escalations consume administrative bandwidth and reflect poorly on institutional efficiency.

\subsection{Financial Misconduct and Bribery Risks}
Manual discretion introduces the risk of:
\begin{itemize}
    \item Informal monetary demands
    \item Favoritism
    \item Unequal enforcement of policies
\end{itemize}
Even isolated incidents undermine trust in the system and damage institutional credibility.

\subsection{Forgery and Technical Exploitation}
Digitally literate students can exploit weaknesses by:
\begin{itemize}
    \item Editing screenshots of VTOP approvals
    \item Using cloned or fake interfaces
    \item Reusing old approvals
\end{itemize}
Supervisors, often without technical training, cannot reliably detect such manipulation.

\subsection{Lack of Auditability}
The current process provides:
\begin{itemize}
    \item No centralized exit logs
    \item No time-stamped verification records
    \item No automated anomaly detection
\end{itemize}
This makes post-incident investigations difficult and weakens accountability.

\subsection{Operational Fragility (Network and Process)}
\begin{itemize}
    \item Verification depends on ad-hoc phone calls or chat confirmations when screenshots are unclear
    \item No offline or degraded-mode strategy during connectivity issues
    \item No rate-limiting or surge-handling approach during peak exits
\end{itemize}

\subsection{Alignment with Project Codebase}
Within this repository, leave creation, verification, and approvals are routed through Flask views in \texttt{app.py}, persisted via MySQL helpers in \texttt{database.py}, and rendered through templates under \texttt{templates/}. PDF generation is handled in \texttt{pdf\_generator.py}. The absence of a secure, single-use QR layer means exit points still rely on manual confirmation.

%-------------------------------------------------
\section{Problem Statement}
There exists a critical gap between digital leave approval systems and physical hostel exit enforcement. The absence of secure, automated verification at the point of exit leads to inefficiency, misuse, student distress, and governance risks. A system is required that enforces hostel leave policies automatically, eliminates human discretion at verification points, and ensures full traceability.

%-------------------------------------------------
\section{Objectives of the Proposed System}
\begin{enumerate}
    \item Eliminate manual, screenshot-based verification at exits
    \item Prevent forgery, impersonation, and reuse of approvals
    \item Reduce operational load on hostel supervisors and proctors
    \item Ensure unbiased and uniform policy enforcement
    \item Create a complete, time-stamped audit trail of hostel exits
    \item Integrate tightly with existing leave workflows in the Flask app
\end{enumerate}

%-------------------------------------------------
\section{Proposed Solution Overview}
The system introduces a \textbf{QR-based digital authorization mechanism} integrated with the existing leave approval workflow. Upon proctor approval, a unique, time-bound, single-use QR token is generated. This token is verified through a centralized backend when scanned at a hostel kiosk. Upon successful verification, a physical permit slip can be printed automatically and the exit recorded.

%-------------------------------------------------
\section{Design Philosophy}
\begin{itemize}
    \item \textbf{Zero Trust on Visual Proof}: Screenshots are never trusted
    \item \textbf{Server-Side Authorization}: All decisions occur centrally
    \item \textbf{Single-Use, Time-Bound Tokens}: Prevent reuse or sharing
    \item \textbf{Auditability by Design}: Every action is logged end-to-end
    \item \textbf{Fail-Safe Defaults}: Deny by default on expired or unverifiable tokens
\end{itemize}

%-------------------------------------------------
\section{System Architecture}
\subsection{Components}
\begin{itemize}
    \item Student Interface (existing Flask templates and routes)
    \item Proctor Approval Module (existing approval routes)
    \item Central Verification Server (Flask endpoint validating QR tokens)
    \item Hostel Kiosk Interface (scanner + minimal UI; can be web-based)
    \item Audit and Logging Database (leveraging existing MySQL schema)
\end{itemize}
Communication occurs via secure REST APIs. Tokens are short-lived and single-use.

\subsection{Alignment with Codebase}
\begin{itemize}
    \item Token generation endpoint added to \texttt{app.py}
    \item Token storage/table added via \texttt{db\_migration.py} or \texttt{update\_schema.py}
    \item PDF embedding of QR in \texttt{pdf\_generator.py}
    \item Verification route for kiosks in \texttt{app.py}
    \item Templates updated under \texttt{templates/} to surface QR download/print
\end{itemize}

%-------------------------------------------------
\section{Feature Inventory (Current and Planned)}
\subsection{Student-Facing}
\begin{itemize}
    \item Account login via email/password; session-based auth in \texttt{app.py}
    \item Leave application form (dates, reason, contact) rendered from \texttt{apply\_leave.html}
    \item Dashboard showing leave history and statuses in \texttt{student\_dashboard.html}
    \item PDF download of approved permission slip once finalized
\end{itemize}

\subsection{Hostel Verifier}
\begin{itemize}
    \item Pending leave queue with verify/reject actions in \texttt{hostel\_verify.html}
    \item Notes capture for verification decisions
    \item Status change to \emph{pending proctor approval} on verify
\end{itemize}

\subsection{Proctor}
\begin{itemize}
    \item List of hostel-verified requests in \texttt{proctor\_dashboard.html}
    \item Approve/reject with remarks; status moves to admin stage on approve
\end{itemize}

\subsection{Admin}
\begin{itemize}
    \item Comprehensive leave list and filtering in \texttt{admin\_leaves.html}
    \item User management in \texttt{admin\_users.html}
    \item System logs view in \texttt{admin\_logs.html}
    \item Finalization step triggers PDF + QR issuance (planned) and marks leave as finalized
\end{itemize}

\subsection{Common/Infrastructure}
\begin{itemize}
    \item Auth routes shared across roles in \texttt{app.py}
    \item Database helpers in \texttt{database.py}; schema bootstrap via \texttt{create\_database.py} and \texttt{init\_db.py}
    \item PDF generation via \texttt{pdf\_generator.py}; QR embedding planned here
    \item Error pages \texttt{404.html}, \texttt{500.html}
\end{itemize}

%-------------------------------------------------
\section{Route Map (Indicative)}
\begin{longtable}{p{0.18\linewidth}p{0.12\linewidth}p{0.58\linewidth}}
Path & Method & Purpose / Role \\ \hline
/ & GET & Landing / role selection \\
/login & GET/POST & Auth for students/hostel/proctor/admin \\
/student/dashboard & GET & Student overview of leaves \\
/student/apply & GET/POST & Submit new leave; validates dates/reason \\
/hostel/verify & GET/POST & Hostel queue to verify or reject \\
/proctor/review & GET/POST & Proctor review of verified requests \\
/admin/dashboard & GET & Admin overview and shortcuts \\
/admin/leaves & GET/POST & Finalize, update, or search leaves \\
/admin/users & GET/POST & Manage user records/roles \\
/admin/logs & GET & View audit logs \\
/leave/<id>/pdf & GET & Generate/download permission slip (calls \texttt{pdf\_generator.py}) \\
/qr/issue & POST & Issue QR token on approval (planned) \\
/qr/verify & POST & Kiosk verification endpoint (planned) \\
\end{longtable}

%-------------------------------------------------
\section{Template Overview}
\begin{itemize}
    \item \texttt{base.html}: Common layout, nav, flashes
    \item \texttt{student\_dashboard.html}: Lists leave history, status badges, actions
    \item \texttt{apply\_leave.html}: Form for dates/reason/contact; server-side validation
    \item \texttt{hostel\_verify.html}: Pending cards/table with approve/reject controls
    \item \texttt{proctor\_dashboard.html}: Queue of hostel-verified requests with remarks field
    \item \texttt{admin\_dashboard.html}: KPIs and quick links
    \item \texttt{admin\_leaves.html}: Search/filter, finalize, export PDF
    \item \texttt{admin\_users.html}: Create/reset roles and credentials
    \item \texttt{admin\_logs.html}: Audit trail display
    \item \texttt{hostel\_login.html}, \texttt{proctor\_login.html}, \texttt{admin\_login.html}, \texttt{student\_login.html}: Role-specific login pages
    \item \texttt{permission\_slip.html}: PDF-friendly permission slip layout (used by generator)
    \item \texttt{404.html}, \texttt{500.html}: Error surfaces with friendly messaging
\end{itemize}

%-------------------------------------------------
\section{PDF and QR Issuance Flow}
\begin{enumerate}
    \item Admin finalizes an approved leave.
    \item System (planned) generates a signed, single-use QR token and stores it with expiry.
    \item \texttt{pdf\_generator.py} renders \texttt{permission\_slip.html}, embedding the QR and human-readable details.
    \item Student downloads or prints the PDF; kiosk scans the QR at exit.
    \item Verification endpoint validates token, marks it used, logs the scan, and optionally prints a kiosk slip.
\end{enumerate}

%-------------------------------------------------
\section{Database Design (Detailed)}
\subsection{Key Tables and Fields}
\begin{longtable}{p{0.22\linewidth}p{0.72\linewidth}}
users & id (PK), name, email (unique), role (student/hostel/proctor/admin), password\_hash, created\_at, updated\_at \\
leaves & id (PK), student\_id (FK users), start\_date, end\_date, reason, contact, status, hostel\_verified\_by/at, proctor\_approved\_by/at, admin\_finalized\_by/at, pdf\_path, created\_at, updated\_at \\
logs & id (PK), actor\_id (FK users), action, leave\_id (nullable FK), metadata (JSON/text), created\_at \\
tokens (planned) & id (PK), leave\_id (FK), token, signature, expires\_at, used\_at, issued\_at, issued\_by, kiosk\_bound (optional) \\
kiosks (planned) & id (PK), name, location, api\_key, status, created\_at \\
\end{longtable}

\subsection{Indexes and Constraints}
\begin{itemize}
    \item Unique index on \texttt{users.email}
    \item Indexes on \texttt{leaves.status}, \texttt{leaves.student\_id}, \texttt{tokens.token}, \texttt{tokens.expires\_at}
    \item Foreign keys to enforce referential integrity (students to leaves, leaves to logs/tokens)
\end{itemize}

%-------------------------------------------------
\section{End-to-End Workflow (Happy Path)}
\begin{enumerate}
    \item Student logs in and submits leave with dates/reason/contact; server validates overlaps and required fields.
    \item Hostel verifier reviews pending queue, verifies or rejects; on verify, status advances.
    \item Proctor reviews verified requests, approves/rejects; on approve, status advances to admin.
    \item Admin finalizes; system issues QR token and generates PDF permission slip.
    \item Student presents QR at kiosk; scan validates token, logs exit, and marks token used.
\end{enumerate}

%-------------------------------------------------
\section{Failure and Edge Cases}
\begin{itemize}
    \item Overlapping leave dates rejected with clear message.
    \item Expired or reused QR: kiosk denies and logs attempt.
    \item Network outage at kiosk: default-deny; optionally queue offline scans for short intervals.
    \item Missing records: routes return 404 with \texttt{404.html}; server errors fall back to \texttt{500.html}.
    \item Role mismatch: RBAC denies access and redirects to login.
\end{itemize}

%-------------------------------------------------
\section{Security and Privacy (Expanded)}
\begin{itemize}
    \item Passwords hashed (e.g., Werkzeug/bcrypt); no plaintext storage.
    \item Sessions: HttpOnly cookies; recommend Secure flag in production.
    \item RBAC enforced per route (student/hostel/proctor/admin separation).
    \item Input validation on forms (dates, contact info, reason length); server-side canonical checks.
    \item SQL safety via parameterized queries/ORM helpers in \texttt{database.py}.
    \item Secrets: \texttt{SECRET\_KEY} and DB creds sourced from environment; never hard-coded in VCS.
    \item Audit logging in \texttt{logs} table for approvals, rejections, scans, and admin actions.
    \item QR tokens: signed, time-bound, single-use; optionally bound to kiosk identity.
\end{itemize}

%-------------------------------------------------
\section{Operations and Deployment}
\subsection{Configuration}
\begin{itemize}
    \item \texttt{DB\_HOST}, \texttt{DB\_USER}, \texttt{DB\_PASSWORD}, \texttt{DB\_NAME}
    \item \texttt{SECRET\_KEY} for Flask sessions
    \item \texttt{TOKEN\_TTL\_MINUTES} for QR validity window
\end{itemize}

\subsection{Setup Steps}
\begin{enumerate}
    \item Install dependencies: \texttt{pip install -r requirements.txt}
    \item Create DB/schema: \texttt{python create\_database.py}, then \texttt{python init\_db.py}
    \item Apply migrations when added: \texttt{python db\_migration.py} or \texttt{python update\_schema.py}
    \item Run app: \texttt{python app.py}
\end{enumerate}

\subsection{Logging and Monitoring}
\begin{itemize}
    \item Application logs via Flask logging (stdout/file)
    \item Audit trail in \texttt{logs} table; filterable by actor, leave, action
    \item Alerts recommended on repeated QR verification failures or kiosk downtime
\end{itemize}

%-------------------------------------------------
\section{Testing and Quality}
\begin{itemize}
    \item \textbf{Unit}: Token creation/validation, date overlap checks, form validation helpers
    \item \textbf{Integration}: End-to-end approval flow (student \rightarrow hostel \rightarrow proctor \rightarrow admin); PDF generation success; QR issue/verify (planned)
    \item \textbf{DB}: Migration tests ensure new token/kiosk tables exist and indexes are applied
    \item \textbf{Performance}: Peak-hour scan simulations to validate latency budgets
    \item \textbf{Security}: Negative tests for expired/reused tokens, role access enforcement
\end{itemize}

%-------------------------------------------------
\section{Performance and Scalability}
\begin{itemize}
    \item Index frequent lookups (token, status, leave\_id, student\_id)
    \item Paginate dashboard lists to reduce payloads
    \item Cache static assets; enable gzip/HTTP compression at proxy
    \item Keep verification endpoint idempotent and lightweight; target sub-200ms responses on LAN
\end{itemize}

%-------------------------------------------------
\section{QR Code Design and Security Model}
\subsection{QR Content Structure}
QR encodes a reference token only, for example:
\begin{verbatim}
LV-YYYYMMDD-XXXX
\end{verbatim}
\subsection{Authorization Principle}
The QR code itself does not grant permission; it references a server-side authorization record.
\subsection{Cryptographic Signing}
Each token is signed using:
\[
\text{Signature} = \mathrm{HMAC}(K_{server}, LeaveID \Vert Timestamp)
\]
This ensures that even copied QR data cannot be forged. Tokens expire after a defined window and are marked used after first successful verification.

%-------------------------------------------------
\section{Backend Verification Logic}
For every scan, the backend validates:
\begin{itemize}
    \item Approval status of the leave
    \item Time validity window
    \item Usage status (single-use)
    \item Authorized kiosk identity (optional binding)
    \item Cryptographic signature integrity
\end{itemize}
Only successful validation permits exit, updates the audit log, and (optionally) prints a slip.

%-------------------------------------------------
\section{Database Design}
Core entities include:
\begin{itemize}
    \item Student Records
    \item Leave Requests
    \item Approval Logs
    \item Exit Transactions (scan events)
    \item Token Store (QR tokens with status and expiry)
    \item Kiosk Authorization Records
\end{itemize}
This structure supports traceability, reporting, and compliance.

%-------------------------------------------------
\section{Technology Stack}
\subsection{Backend}
\begin{itemize}
    \item Python (Flask)
    \item MySQL (per existing project)
    \item HMAC/JWT signing for tokens
\end{itemize}
\subsection{Kiosk Interface}
\begin{itemize}
    \item Web-based touch interface
    \item QR scanning module (camera or USB scanner)
    \item Thermal printer integration (simulated or real)
\end{itemize}

%-------------------------------------------------
\section{Operational Considerations}
\begin{itemize}
    \item \textbf{Network Dependency}: Provide cached validation for very short outages with rapid replay; default-deny if uncertain
    \item \textbf{Throughput}: Use short, idempotent verification calls; index by token and status
    \item \textbf{Monitoring}: Log all scans with timestamps and kiosk IDs; alert on repeated failures
\end{itemize}

%-------------------------------------------------
\section{Administrative Benefits}
\begin{itemize}
    \item Reduced supervisor workload and discretion
    \item Faster student processing during peaks
    \item Transparent, queryable exit records
    \item Lower risk of forgery and misconduct
\end{itemize}

%-------------------------------------------------
\section{Limitations and Practical Constraints}
\begin{itemize}
    \item Network dependency at kiosks
    \item Initial hardware setup and training costs
    \item Policy alignment and change management required
\end{itemize}

%-------------------------------------------------
\section{Future Scope}
\begin{itemize}
    \item RFID or biometric integration
    \item Analytics dashboard for hostel authorities
    \item Emergency override mechanisms with audit
    \item Integration with campus attendance systems
\end{itemize}

%-------------------------------------------------
\section{Testing Strategy}
\begin{itemize}
    \item Unit tests for token creation/validation (add to \texttt{test\_login.py} or new module)
    \item Integration tests for scan workflow (add to \texttt{test\_admin\_features.py})
    \item DB migration tests to ensure token tables are created and indexed
    \item Performance checks for peak scan volumes
\end{itemize}

%-------------------------------------------------
\section{Performance Considerations}
\begin{itemize}
    \item Index frequent lookups (token, status, leave\_id)
    \item Paginate dashboard lists
    \item Cache static assets; enable gzip at the web server/proxy
\end{itemize}

%-------------------------------------------------
\section{Conclusion}
The Smart QR-Based Hostel Leave Verification System addresses a long-standing administrative and governance challenge within university hostels. By removing human discretion at the verification stage and enforcing policy through secure digital mechanisms, the system improves efficiency, transparency, and student welfare while strengthening institutional control. The design aligns with the existing Flask codebase and MySQL schema, enabling incremental adoption with focused changes to routes, schema, and PDF generation.

\appendix
\section{Sample Config Snippet}
\begin{lstlisting}[language=Python,caption={Config example for DB and secret key}]
DB_CONFIG = {
    "host": "localhost",
    "user": "user",
    "password": "password",
    "database": "leave_db",
}
SECRET_KEY = "replace-me"
TOKEN_TTL_MINUTES = 120
\end{lstlisting}

\section{Example Route Stub}
\begin{lstlisting}[language=Python,caption={Leave submission with QR issuance}]
@app.route("/apply", methods=["GET", "POST"])
def apply_leave():
    # authenticate user, validate form, check overlaps, insert leave
    # on approval path, issue QR token and store signed reference
    ...
\end{lstlisting}

\section{Example State Transitions}
\begin{tabularx}{\linewidth}{p{0.28\linewidth}X}
\textbf{From} & \textbf{To / Trigger} \\
\hline
Draft & Pending Hostel Verification (student submits) \\
Pending Hostel Verification & Rejected (hostel rejects) \\
Pending Hostel Verification & Pending Proctor Approval (hostel verifies) \\
Pending Proctor Approval & Rejected (proctor rejects) \\
Pending Proctor Approval & Pending Admin Finalization (proctor approves) \\
Pending Admin Finalization & Approved/Finalized (admin finalizes + QR + PDF) \\
Approved & Used (kiosk scan consumes token) \\
Any & Cancelled (student/admin cancel, optional) \\
\end{tabularx}

\end{document}
